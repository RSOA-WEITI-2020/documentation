Rozdział ten przybliża wymagania funkcjonalne oraz niefunkcjonalne, które aplikacja wraz z systemem miały spełnić.
Zapewnia też krótkie omówienie każdego z przedstawionych wymagań.

\section{Wymagania funkcjonalne}

System jest dość prosty w związku z czym nie posiada wielu wymagań funkcjonalnych.
Wymagania te można zgrupować w ramach dwóch zagadnień. 

Wymagania związane z \textbf{zarządzaniem użytkownikami} dotyczą procedur, które 
muszą zostać zapewnione w związku z RODO, ale też funkcjonalności pozwalających na
podstawowe operacje w związku z tworzeniem i edycją użytowników.

Wymagania związanie z \textbf{zarządzaniem zgłoszeniami} dotyczą bezpośrednio funkcjonalności 
związanych z możliwością tworzenia i sprawdzania stany zgłoszeń użytkownika.
Część funkcjonalności została dodana, tak jak w przypadku zarządzania użytkownikami, ze względu
na zgodność z RODO.

\subsection{Zarządzanie użytkownikami}
\label{sec:user_func}

\subsubsection{Tworzenie nowego konta}

Użytkownik musi mieć możliwość stworzenia dla siebie konta, którego później będzie używał do
korzystania z aplikacji. Konto musi być unikatowe według wybranego identyfikatora (np. adresu email).
Przy rejestracji konta użytkownik podaje dane billingowe, które podlegają pod ochronę na mocy RODO. 

\subsubsection{Usunięcie konta użytkownika}

Użytkownik musi mieć możliwość usunięcia swojego konta z poziomu aplikacji.
Na takie konto nie będzie można się już zalogować. 
Można będzie też założyć nowe konto z identyfikatorem takim samym jak to usunięte. 

\subsubsection{Usunięcie konta wraz z wszystkimi powiązanymi danymi}

Użytkownik musi mieć możliwość wystosowania prośby o lub bezpośrednie usunięcie
swojego konta z wszystkimi powiązanymi z nim danymi np. historia zgłoszeń, wyniki obliczeń.
Możliwość ta jest wymagana na mocy RODO.

\subsubsection{Pobranie wszystkich danych związanych z kontem}

Użytkownik musi mieć możliwość pobrania wszystkich informacji przechowywanych o nim
w bazach danych systemu. Wchodzą w to dane podane przy rejestracji jak i
historia wszystkich zgłoszeń użytkownika. 
Funkcjonalność ta jest wymagana przez RODO.

\subsection{Zarządzanie zgłoszeniami}

\subsubsection{Dodanie zgłoszenia}

Użytkownik musi mieć możliwość dodania nowego zgłoszenia do symulacji.
Można to zrobić przez bezpośrednie wpisanie kodu aplikacji w języku \textbf{qasm}
w prostym edytorze zapewnionym przez aplikację lub wysłanie pliku z kodem.
Aplikacja klienta nie sprawdza poprawności semantycznej i składniowej kodu źródłowego
przed wysłaniem go do symulacji.

\subsubsection{Przeglądanie zgłoszeń}

Użytkownik musi mieć możliwość przeglądania wszystkich jego zgłoszeń.
Dotyczy to zarówno zgłoszeń zakończonych jak i tych obecnie trwających.
Aplikacja powinna w czytelny sposób pokazywać jaki jest stan każdego zgłoszenia
(zakończone, w trakcie realizacji, zakończone z błędem).
Aplikacja powinna pokazywać dodatkowe informacje o zgłoszeniach ułatwiające
ich identyfikację (możliwość wyświetlenia kodu źródłowego, czas rozpoczęcia zgłoszenia,
czas zakończenia zgłoszenia)

\subsubsection{Pobranie historii zgłoszeń}

Jak wspomniano w sekcji \ref{sec:user_func} użytkownik powinien mieć możliwość
pobrania historii swoich zgłoszeń jako oddzielnego pliku.

\subsubsection{Przeglądanie wyniku zakończonego zgłoszenia}

Jeżeli zgłoszenie zostało zakończone użytkownik powinien mieć możliwość:
\begin{itemize}
    \item zobaczenia wyniku symulacji, jeżeli zgłoszenie zakończyło się sukcesem;
    \item zobaczenia wiadomości błędu, jeżeli zgłoszenie zakończyło się błędem;
\end{itemize}

\section{Wymagania niefunkcjonalne}

\subsection{Niezawodność}

\subsection{Skalowalność}

\subsection{Bezpieczeństwo}

\subsection{Responsywność}

\subsection{Rozmiar przechowywanych danych}

Użytkownik może wysyłać do przetworzenia pliki lub też samemu wpisywać kod
źródłowy programy do analizy. Konieczne jest narzucenie limitu na to jakiej 
wielkości i też ile danych użytkownik może wprowadzić do systemu.

